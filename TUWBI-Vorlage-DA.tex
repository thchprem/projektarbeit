%% LaTeX template for diploma thesis at the 
%% department of civil engineering of TU Wien
%% Following files have to be in the same directory:
%% TUWBIDA.sty and tu-bi-2009.jpg
%% 
%% created by Christian Schranz and Sebastian Pech
%% EDV-Zentrum Bauingenieurwesen
%% date: 2016-10-14
%% tested for pdflatex
%%
\documentclass[11pt,twoside=true]{scrreprt}
%\documentclass[11pt,oneside]{scrreprt}
%
%% consider that the input encoding of all should be utf8 
%% if you use the follwing setting, otherwise change it
\usepackage[utf8]{inputenc}

\usepackage{TUWBIDA}
%% this style includes already some packages which have been very useful 
%% in the last diploma or doctoral theses
%%
%% fontenc[T1], lmodern, microtype, babel[englisch,naustrian], graphicx, 
%% geometry with all margins or areaset (choose what you like)
%% mathtools, amssymb, nicefrac, siunitx[, as decimal marker], booktabs,
%% url, xcolor[table], textcomp, marvosym, pifonts, pdfpages, ragged2e, 
%% tabularx, longtable, csquotes, eurosym, eso-pic, enumitem, todonotes, 
%% dcolumn, multirow, setspace, listings
%% scrlayer-scrpage with header/footline

\raggedbottom 
%% prevents the expansion of the text till the end of the page (if you like)
%\setcapindent{0em} %% influences captions layout

%% ===== title information ==============
%% The title page is designed in the style file TUWBIDA.sty
%% Only the following information has to be provided
%% Every diploma thesis has an english and a german title
%%
%% german title
\titleGER{Leistungs- und Kostenermittlung im maschinellen Tunnelbau}
%% english title
\titleENG{The big river}
%% author with correct academic titles, if necessary
\author{\textbf{Theresa Prem}}
%% provide the gender for correct title page: M...male, F...female
\gender{F}
%% student id (or Matrikelnummer)
\MatrNr{1025020}
%%
%% supervising professor and assistents with correct titles 
%% use \\[2ex] to separate them
\supervisor{Univ.Prof. Dipl.-Ing. Dr.techn. \textbf{Gerald Goger}
\\[2ex]
Dipl.-Ing. \textbf{Melanie Piskernik}}
%% supervising institute
\institute{Institut für interdisziplinäres Bauprozessmangament -- Bereich Baubetrieb und Bauwirtschaft\\
Technische Universität Wien \\
Karlsplatz 13/234-1, A-1040 Wien}
%% date
\date{Oktober 2016}

%% use biblatex and biber for bibliography
%% customize options as you like
%% style=numeric-comp ... [1]
%% style=authoryear ... Mang 1998 / use \textcite{} ... Mang (1998)
%%
\usepackage[style=numeric-comp,backend=biber,maxcitenames=1]{biblatex}
\ExecuteBibliographyOptions{%
  firstinits=true,maxbibnames=99}%
\DefineBibliographyStrings{ngerman}{andothers={et\;al\adddot}}
\addbibresource{Literatur.bib}


%% ===== additional packages to the ones already loaded ==============

%% ===== additional settings =============
\setcounter{secnumdepth}{3}


\begin{document}       %% start of the document
\maketitle             %% places the title with above information

\cleardoublepage
\selectlanguage{ngerman}  %% for german abstract
\chapter*{Kurzfassung}
      %% german abstract (about 1 page)

%\cleardoublepage
\selectlanguage{ngerman} 
\pagestyle{scrheadings} 

\tableofcontents

%% now include all chapter files -- use the include-command (takes care
%% that all not yet output floats are output before the new file)
%%
%\include{01-introduction}
%\include{02-Grundlagen}
%etc.
\cleardoublepage
\selectlanguage{ngerman}  %% for german abstract
\chapter{Einleitung}
Der maschinelle Tunnelbau, auch kontinuierlicher Vortrieb (KV), hat sich in den letzten Jahren aufgrund der technologischen Neuheiten enorm entwickelt. Allerdings müssen die Randbedingungen stimmen um ein wirtschaftliches Ergebnis, speziell im Vergleich zum zyklischen Vortrieb (ZV), zu erzielen.
\\
Vorteile:
\begin{itemize}
\item sehr hohe Vortriebsleistung möglich
\item exaktes Ausbruchsprofil
\item niedriger Personalaufwand
\item gute Arbeitsbedingungen und Sicherheit
\item Mechanisierung und Automatisierung des Vortriebs
\end{itemize}
Nachteile:
\begin{itemize}
\item bessere geologische Vorauserkundungen und Informationen als beim zyklischen Vortrieb
\item hohe Investitionskosten rechnen sich erst bei längeren Strecken
\item lange Vorlaufzeit für Planung und Bau der Maschine
\item Kreisprofil
\item Limitierung der möglichen Kurvenradien und Aufweitungen
\item Detaillierte Planung notwendig
\item Anpassung an unterschiedliche Gesteinsvorkommen und hohen Wassereintritt nur bedingt möglich
\item Anlieferung der Maschine zum Tunnel
\end{itemize}
Es ist eine sehr gute Planung nötig um die Vorteile optimal zu nutzen und die Nachteile bestmöglich zu minimieren. Nur wenn diese Punkte beachtet werden ist ein effizienter Einsatz möglich.
\
Einteilung der Vortriebssysteme (nach DAUB):
\paragraph{Tunnelbohrmaschinen (TBM)}
Das Grundprinzip der TBM besteht darin, dass eine Verspannung mit dem standfesten Gebirge erzeugt wird und Hydraulikzylinder den Bohrkopf an die Ortsbrust drücken. Durch die Rotation des Bohrkopfs und den Druck der Zylinder lösen die Disken das anstehende Gebirge. Diese Chips (Bohrklein) werden durch Räumerkammern auf das Maschinenförderband transportiert und weiter Obertage befördert. Die Kolbenlänge der Zylinder bestimmt den maximalen Hub (vergleichbar mit der Abschlagslänge beim zyklischen Vortrieb). Nachdem ein Hub aufgefahren wurde, wird die Maschine umgesetzt um von neuem mit dem Bohrvorgang zu beginnen. Parallel zum Bohren wird der Tunnel mit Tübbingen ausgekleidet. Ein Tübbingring besteht aus ca. 6 Steinen (je nach System) und hat die Breite eines Hubs. Die Auskleidung übernimmt die Sicherung des Tunnels. Aufgrund der hohen Ansprüche an die Tübbinge (gute Betonqualität usw.) ist es meist nicht notwendig zusätzlich eine Innenschale zu betonieren, d.h. der Tunnelausbau ist einschalig.

\begin{itemize}
\item Einsatz im standfesten Festgestein
\item aktive Stützung der Ortsbrust nicht notwendig (technisch auch nicht möglich)
\item voller Kreisquerschnitt wird aufgefahren
\end{itemize}
\begin{enumerate}
\item TBM ohne Schild (TBM-O): Maschine verspannt sich radial mit Gripperplatten gegen die Ausbruchslaibung und bringt so an den Anpressdruck auf den Bohrkopf auf
\item Aufweitungsmaschine ohne Schild (TBM-A): vergrößern einen zuvor hergestellten Pilotstollen, im Fall von Störzonen können Maßnahmen vom Pilotstollen durchgeführt werden
\item TBM mit Schild (TBM-S): Einsatz im Festgestein mit geringer Standzeit bzw. nachbrüchigem Fels - Maschine stützt sich am Schildmantel ab, der auch zum Schutz des Ausbaus (Tübbinge) dient
\end{enumerate}
\paragraph{Schildmaschinen (SM)}
\begin{itemize}
\item Einsatz im Lockergestein, auch im Grundwasser
\item Voll- oder Teilschnittabbau (je nach Maschinentyp)
\item Stützung der Ortsbrust und des Hohlraums notwendig durch mechanische Stützung, Druckluftbeaufschlagung, Flüssigkeitsstützung oder Erddruckstützung
\end{itemize}      %% german abstract (about 1 page)

\cleardoublepage
\selectlanguage{ngerman}  %% for german abstract
\chapter{Leistung}
\label{leistung}
Die Leistungsermittlung stellt eine wesentliche Basis für die weitere Kostenabschätzung eines bautechnischen Projektes dar. Besonders im Tunnelbau ist diese Abschätzung der Vortriebsleistung von großer Bedeutung. Die drei Haupteinflussfaktoren sind laut \parencite{thuro2011}:
\begin{itemize}
\item Gestein und Gebirge: geologisch und felsmechanische Parameter
\item Maschinenparameter: TBM-Technik
\item Baubetrieb: Logistik, Bedienung und Wartung
\end{itemize}
%...
Die Umrechnung von Netto- auf Bruttovortriebsleistung wird über den sogenannten Ausnutzungsgrad gemacht.

\section{Penetration}
\label{penetration}
Penetration beschreibt die Eindringtiefe der Bohrwerkzeuge je Umdrehung des Bohrkopfes. Die typische Einheit ist Millimeter pro Umdrehung [mm/U]. 
Durch die Rotation und die gleichmäßige Andruckkraft (i.d.R. 200kN) werden die Rollenmeissel in konzentrischen Bahnen an die Ortsbrust gedrückt. lösen die sogenannten Chips ab.
\\In den folgenden Punkten werden die gängisten Berechnungsmodelle aufgezeigt.
\paragraph{Gesteinslösevorgang} 

%Bild schema_ausbruch

\subsection{Prognosemodell nach Gehring}
\label{prognose_gehring}
Dieses empirische Modell ist weitverbreitet aufgrund seiner unkomplizierten Anwendung. Eingang finden die einaxiale Druckfestigkeit und die Gefügeeigenschaften. Die Druckfestigkeit ist die einzig notwendige Laboruntersuchung. Falls diese nicht möglich ist, kann sie - gleich wie die Gefügeeigenschaften - von den Geologen abgeschätzt werden. Somit kann mit diesem Berechnungsmodell eine Penetration zu einem sehr frühen Zeitpunkt des Projekts gerechnet werden.\parencite{leitner2004}
\
Den Zerspaltungsvorgang beschreibt Gehring in 4 Phasen:
\begin{enumerate}
\item[] Eindringen der Schneidrolle und erzeugen der Zermalmungszone
\item[] Bildung von Zugrissen aus der Zermalmungszone
\item[] Spanbildung nach erreichen des überkritischen Bruchzustandes
\item[] Lösen des Spans und Spannungsabbau
\end{enumerate}

Annahmen für die Eingangsparameter: Schneidbahnabstand s = 80mm, Diskendurchmesser 17Zoll (430mm), und die Andruckkraft je Diske von 200kN

\begin{equation}
\label{eg:gehring}
p=4*F_{N}/\sigma_{d}*(k_{1}*k_{2}*...)
\end{equation}
mit
\begin{description}[labelindent=1cm]
\item[] p ... Penetration [mm/rev]
\item[] $F_{N}$ ... mittlere Andruckkraft (200kN)
\item[] $\sigma_{d}$ ... einaxiale Druckfestigkeit [N/$mm^{2}$]
\item[] k ... Korrekturfaktor
\end{description}

%
Die Berechnung der Korrekturfaktoren würde den Umfang dieser Arbeit überschreiten und somit kommt die vereinfachten Formel zum tragen. 
\
\begin{equation}
\label{eg:gehringeinfach}
P=4*200/\sigma_{d}
\end{equation}

\subsection{Penetrationsermittlung der NTNU Trondheim}
\label{penetration_ntnu}
In diesem empirischen Modell ist die einaxiale Druckfestigkeit in erster Linie kein wesentlicher Parameter. Es wurde für nordische Gesteine entwickelt und anschließend mit Standard-Gesteinsparametern verknüpft, um es für verschiedene Gesteine anwendbar zu machen. Somit ist es die meist angenommene empirische Methode in Europa um Prognosen zu erstellen.
Die notwendigen Untersuchungen sind zum einen die Bohrbarkeit des Gesteins (liefert DRI) und zum anderen die Gefügeeigenschaften des Gebirges (Abschätzung der Trennflächenorientierung und -abstände). Weiters werden Maschinendaten berücksichtigt wie Form, Anordnung und Größe der Werkzeuge, sowie die Rollengeschwindigkeit, verfügbare Andruckkraft und Maschinendynamik.

\paragraph{äquivalenter Gebirgsfaktor}
\begin{equation}
\label{eg:gebirgsfaktor}
k_{ekv}=k_{s-tot}*k_{DRI}*k_{por}
\end{equation}
%Diagramme Abb. 39, 40

\paragraph{äquivalente Andruckkraft}
\begin{equation}
\label{eg:andruckkraft}
M_{ekv}=M_{B}*k_{d}*k_{a}
\end{equation}

\paragraph{Penetration}
\begin{equation}
\label{eg:penetration}
i_{0}=(M_{ekv}/M_{1})^{b}
\end{equation}
%Abb. 43

\subsection{Prognosemodell der Colorado School of Mines}
\label{prognose_csm}
Dieses Modell dient in erster Linie zur Effizienzoptimierung und ist sinnvoll in seiner Anwendung wenn es um die Optimierung des Bohrkopfdesigns geht. Eingangswerte sind das Bohrkopfprofil und die Gesteinseigenschaften.
\paragraph{resultierende Schneidkraft}
\begin{equation}
F_{t}=P°*\Phi*R*T/(1+\Psi)
\end{equation}

\begin{equation}
\Phi=\arccos((R-p)/R)
\end{equation}

\section{Werkzeugverbrauch}
\label{werkzeugverbrauch}


\section{Baubetriebliche Modellierung}
\label{baubetr_modell}
      %% german abstract (about 1 page)

\cleardoublepage
\selectlanguage{ngerman}  %% for german abstract
\chapter{Logistik}
Die Logistik ist ebenfalls ein wesentlicher Bestandteil um wirtschaftlich gute Ergebnisse zu erzielen. Das bewältigen der großen Massen, gute Personaleinsatzplanung ist von großer Bedeutung und soll exemplarisch in den nächsten Punkten erläutert werden.
\section{Material}
Der Transport des Materials kann radgebunden, gleisgebunden oder über Förderbänder geschehen.
Die Geschwindigkeit des Bandes wird mit ..km/h angenommen.
\section{Personal}
Bei Tunnelbauprojekten mit TBM-Vortrieb ist der 4/3-Dekadenbetrieb üblich.
%Bild einfügen von VO-Unterlagen
Somit ergibt sich die monatliche Schichtarbeitszeit:
\begin{equation}
AZ_{Schicht}=3 Drittel*8 h/AT*7 AT*4,33Wo/Mo=727,44 h/Mo
\end{equation}

 

\cleardoublepage
\selectlanguage{ngerman}  %% for german abstract
\chapter{Kosten}
Wirtschaftliche 
\section{Gerätekosten}
klöaskfdöakdfsjkdösa
\section{Personalkosten} 

\cleardoublepage
\selectlanguage{ngerman}  %% for german abstract
\chapter{Anwendungsbeispiel}
Wirtschaftliche 
\section{Eingangswerte}
klöaskfdöakdfsjkdösa 

\cleardoublepage
\selectlanguage{ngerman}  %% for german abstract
\chapter{Zusammenfassung}
asfdjklö 

%% insert bibliography, adds chapter heading with chapter number
\printbibliography

\end{document}