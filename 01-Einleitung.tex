\chapter{Einleitung}
Der maschinelle Tunnelbau, auch kontinuierlicher Vortrieb (KV), hat sich in den letzten Jahren aufgrund der technologischen Neuheiten enorm entwickelt. Allerdings müssen die Randbedingungen stimmen um ein wirtschaftliches Ergebnis, speziell im Vergleich zum zyklischen Vortrieb (ZV), zu erzielen.
\\
Vorteile:
\begin{itemize}
\item sehr hohe Vortriebsleistung möglich
\item exaktes Ausbruchsprofil
\item niedriger Personalaufwand
\item gute Arbeitsbedingungen und Sicherheit
\item Mechanisierung und Automatisierung des Vortriebs
\end{itemize}
Nachteile:
\begin{itemize}
\item bessere geologische Vorauserkundungen und Informationen als beim zyklischen Vortrieb
\item hohe Investitionskosten rechnen sich erst bei längeren Strecken
\item lange Vorlaufzeit für Planung und Bau der Maschine
\item Kreisprofil
\item Limitierung der möglichen Kurvenradien und Aufweitungen
\item Detaillierte Planung notwendig
\item Anpassung an unterschiedliche Gesteinsvorkommen und hohen Wassereintritt nur bedingt möglich
\item Anlieferung der Maschine zum Tunnel
\end{itemize}
Es ist eine sehr gute Planung nötig um die Vorteile optimal zu nutzen und die Nachteile bestmöglich zu minimieren. Nur wenn diese Punkte beachtet werden ist ein effizienter Einsatz möglich.
\
Einteilung der Vortriebssysteme (nach DAUB):
\paragraph{Tunnelbohrmaschinen (TBM)}
Das Grundprinzip der TBM besteht darin, dass eine Verspannung mit dem standfesten Gebirge erzeugt wird und Hydraulikzylinder den Bohrkopf an die Ortsbrust drücken. Durch die Rotation des Bohrkopfs und den Druck der Zylinder lösen die Disken das anstehende Gebirge. Diese Chips (Bohrklein) werden durch Räumerkammern auf das Maschinenförderband transportiert und weiter Obertage befördert. Die Kolbenlänge der Zylinder bestimmt den maximalen Hub (vergleichbar mit der Abschlagslänge beim zyklischen Vortrieb). Nachdem ein Hub aufgefahren wurde, wird die Maschine umgesetzt um von neuem mit dem Bohrvorgang zu beginnen. Parallel zum Bohren wird der Tunnel mit Tübbingen ausgekleidet. Ein Tübbingring besteht aus ca. 6 Steinen (je nach System) und hat die Breite eines Hubs. Die Auskleidung übernimmt die Sicherung des Tunnels. Aufgrund der hohen Ansprüche an die Tübbinge (gute Betonqualität usw.) ist es meist nicht notwendig zusätzlich eine Innenschale zu betonieren, d.h. der Tunnelausbau ist einschalig.

\begin{itemize}
\item Einsatz im standfesten Festgestein
\item aktive Stützung der Ortsbrust nicht notwendig (technisch auch nicht möglich)
\item voller Kreisquerschnitt wird aufgefahren
\end{itemize}
\begin{enumerate}
\item TBM ohne Schild (TBM-O): Maschine verspannt sich radial mit Gripperplatten gegen die Ausbruchslaibung und bringt so an den Anpressdruck auf den Bohrkopf auf
\item Aufweitungsmaschine ohne Schild (TBM-A): vergrößern einen zuvor hergestellten Pilotstollen, im Fall von Störzonen können Maßnahmen vom Pilotstollen durchgeführt werden
\item TBM mit Schild (TBM-S): Einsatz im Festgestein mit geringer Standzeit bzw. nachbrüchigem Fels - Maschine stützt sich am Schildmantel ab, der auch zum Schutz des Ausbaus (Tübbinge) dient
\end{enumerate}
\paragraph{Schildmaschinen (SM)}
\begin{itemize}
\item Einsatz im Lockergestein, auch im Grundwasser
\item Voll- oder Teilschnittabbau (je nach Maschinentyp)
\item Stützung der Ortsbrust und des Hohlraums notwendig durch mechanische Stützung, Druckluftbeaufschlagung, Flüssigkeitsstützung oder Erddruckstützung
\end{itemize}